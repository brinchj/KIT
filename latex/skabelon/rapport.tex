\documentclass[10pt,a4paper,danish]{article}


%% Indlæs ofte brugte pakker
\usepackage{amssymb}
\usepackage[danish]{babel}
\usepackage[utf8]{inputenc}
\usepackage{listings}
\usepackage{microtype}


%% Opsæt indlæsning af filer
\lstset{
  language=Python,
  extendedchars=\true,
  inputencoding=utf8,
  linewidth=\textwidth, basicstyle=\small,
  numbers=left, numberstyle=\footnotesize,
  tabsize=2, showstringspaces=false,
  breaklines=true, breakatwhitespace=false,
}


%% Titel og forfatter
\title{MIN TITEL \\ Grundlæggende Datalogi}
\author{MIT NAVN}

%% Start dokumentet
\begin{document}

%% Vis titel
\maketitle
\newpage

%% Vis indholdsfortegnelse
\tableofcontents
\newpage



%% HER STARTER MIN AFLEVERING



\section{Indtroduktion}
Jeg har valgt at lave det og det fordi sådan og sådan..


\section{Analyse}
I mit produkt skal man kunne sådan og sådan, men ikke det her for det
ville tage for lang tid.

Når I har valgt hvilken type webapplikation I vil lave, er det næste
at analysere og afgrænse problemstillingen. Hvad er det konkret
opgaven går ud på? Hvad skal den webapplikation I laver
kunne?

Eksempelvis ved en social-netværks-applikation kan dette være
at brugere skal kunne oprette profile med de-og-de persondata, samt at
der skal holdes styr på relationer mellem brugerene - og at der
derudover ikke skal implementeres andre ting. Tag eventuelt
udgangspunkt i use-cases, ie. hvordan skal applikationen bruges, og
hvilke formularer programmet skal præsentere for brugeren.

\subsection{Punkt under analyse}
I \LaTeX\ laver man undersektioner ved brug af "`$\backslash$subsection"'
kommandoen. Alle kommandoer starter med "`$\backslash$"'. Bemærk, at min
undersektion automatisk er kommet med i indholdsfortegnelsen. Jeg kan
også lave en fin lille fodnote\footnote{smart ikke?}.

\section{Design}
For at mit program kan virke, skal der være en database med de og de
tabeller. De skal indeholde...

Hvilke tabeller skal der være i databasen, med hvilke kolonner? Hvad
er der af væsentlig funktionalitet i programmet og hvordan skal det
opbygges?

\subsection{Punkt under design}
Endnu et eksempel på en undersektion.


\section{Implementation}
Det er pænt at lave en undersektion for hver fil/funktionalitet der
præsenteres. Det er også velset, at lægge selve kildekoden i et bilag.

\subsection{Login}
Login gør sådan og sådan og virker på den her måde.
Se bilag \ref{login.py} for kildekoden.

\subsection{Anden funktionalitet}
Noget om "`anden funktionalitet"'. Kildekode ligger i bilag...


\section{Afprøvning}
Og til sidst afprøvning, med test af om
programmet opfylder det, som I i analysen er kommet frem til at det
bør kunne - dette giver også lejlighed til at reflektere over den
løsning I har lavet.

Hvad har i gjort for at afprøve programmet? Fik i det forventede
resultat? Fandt i fejl?


\section{Konklusion}
Gik det godt? Var der problemer? Noget du ikke nåede eller som ikke
virker? Noget der var svært?


%% Her under følger diverse bilag
\appendix


\section{Kildekode}
\subsection{login.py}
\label{login.py} % vi kan referere til dette afsnit med \ref{login.py}

\lstinputlisting{login.py}


\end{document}

